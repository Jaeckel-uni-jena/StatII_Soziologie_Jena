% \documentclass[a4paper,12pt]{article}
% \usepackage[utf8]{inputenc}
% \usepackage[ngerman]{babel} 
% %opening
% \title{}
% %\author{}
% \date{}
%\documentclass[a4paper,fontsize=16pt]{article}
\documentclass[a4paper,fontsize=13pt]{scrartcl}
\usepackage[utf8]{inputenc}
\usepackage{amsmath}
\usepackage{mathtools}
\usepackage{hyperref}
\usepackage[ngerman]{babel} 
\usepackage{textcomp} 
\usepackage[margin=1.55cm
  %,showframe% <- only to show the page layout
]{geometry}
\usepackage{rotating}
\usepackage{booktabs} 
\usepackage{lscape}
\usepackage{float}
%opening
\title{Induktive Statistik}
%\subtitle{Simulation und Auswertung der simulierten Daten in Übung und Tutorium mit Hilfe von \texttt{R}}
%\author{Mariana Nold}
\date{\vspace{-10ex}}
\begin{document}

\maketitle
\begin{enumerate}
   \item{ \textbf{Welche Aufgaben hat die induktive Statistik?}
   \begin{itemize}
      \item{Wiederholung: Wozu braucht man Statistik?}
      \item{Beispiel:  (un)fairer Würfel}
      \item{Simulation des Zufallsexperiment}
      \item{Wiederholung: absolute und relative Häufigkeit}
      \item{Definition: Diskrete Zufallsvariable und Wahrscheinlichkeitsverteilung}
   \end{itemize}
   }
   \item{ \textbf{Schätzer für Wahrscheinlichkeit}
     
       \begin{itemize}
         \item{Die Binomialverteilung: Erwartungswert und Varianz}
         \item{Intuitiver Schätzer für die Wahrscheinlichkeit eine sechs zu würfeln}
         \item{Verteilung des Mittelwerts mit wachsender Anzahl von Versuchen}
         \item{Mittelwert als Schätzer}
         \item{Intuitives Verständnis für das Gesetz der großen Zahlen}
         \end{itemize}
      } 
 
%      \item{ \textbf{Verteilung des Mittelwerts mit wachsender Anzahl von Versuchen}
%        \begin{itemize}
%           \item{Mittelwert als Schätzer} % 
%           \item{Intuitives Verständnis für Gesetz der großen Zahlen}
%         \end{itemize}
%    }
   
    \item{\textbf{Diskrete Verteilungen}
     \begin{itemize}
          \item{Beispiele aus den aktuellen Nachrichten} % 
          \item{Poissonverteilung: Anzahl der Flüchtlinge pro Tag}
          \item{Binomialverteilung: Anteil der Kinder die in Armut leben}
     \end{itemize}
   }
   
      \item{\textbf{Kreuztabellen}
     \begin{itemize}
          \item{Vierfedertafel: stochastische Unabhängigkeit und Kausalität} % 
          \item{Beispiel: Arbeitslose sind häufiger krank. Verursacht Arbeitslosigkeit die Krankheit, oder
              werden Kranke häufiger arbeitslos?}
          \item{Vierfedertafel: relatives Risiko, Odds und Odds Ratio}
          \item{Beispiel zur Drittvariablenkontrolle: Bestehen der Klausur, Besuch
          des Tutoriums und Angst vor Statistik}
          \item{Kreuztabellen: $\chi^{2}-$ Unabhängigkeitstest}%$\Chi$ }
     \end{itemize}
   }
  
     
     \item{\textbf{Grundlage des statistischen Testens }
     \begin{itemize}
       \item{Normalverteilung und Mittelwert}
       \item{Tests über den Mittelwert}
       \item{Konfidenzintervall für den Mittelwert }
       \item{Zusammenhang der Interpretation von Konfidenzintervall und Test}
     \end{itemize}
   }
     
      \item{\textbf{Varianzanalyse}
    \begin{itemize}
      \item{Vergleich von Mittelwerten in zwei Gruppen}
      \item{Vergleich von Mittelwerten in drei und mehr Gruppen}
    \end{itemize}
  }
  
   \item{\textbf{Bivariate lineare Regression}
     \begin{itemize}
          \item{Kovarianz und Korrelation} % 
          \item{Binäre Einflussgrößen}
     \end{itemize}
   }
   
   \item{\textbf{Multivariate lineare Regression}
     \begin{itemize}
        %  \item{Kovarianz und Korrelation} % 
          \item{Interpretation von Schätzern, Konfidenzintervallen und Tests}
          \item{Interpretation der Regressionsgeraden und Konfidenzintervall der Regressionsgeraden}
     \end{itemize}
   }
%  \item{\textbf{Logistische Regression}
%      \begin{itemize}
%        \item{Anwendungsbeispiele }
%        \item{KI und Test für Effekte}
%      \end{itemize}
%    }
  
 % \item{\textbf{Wiederholung}}
  
\end{enumerate}

% Aufgaben der deskrip. Statistik:

% ist es, die in den Daten einer Stichprobe enthaltene relevante Information in Tabellen, Grafiken und statistischen 
% % Maßzahlen übersichtlich und in einem der Fragestellung angemessenen Format zusammenzufassen.

% In der Drittvariablenkontrolle versucht man
% Effekte zwischen zwei Variablen unter
% Konstanthaltung (Aussschaltung) der
% anderen ins Modell aufgenommenen
% Variablen zu berechnen

\end{document}
