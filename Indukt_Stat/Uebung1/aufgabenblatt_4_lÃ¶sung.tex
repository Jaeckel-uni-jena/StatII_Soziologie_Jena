\documentclass[11pt]{article}
\pagestyle{empty}
%\usepackage[latin1]{inputenc}
\usepackage[utf8]{inputenc}
\usepackage{a4wide}
\usepackage{amsmath}
\usepackage{amssymb}
\usepackage{amsthm}
\usepackage{german}
\usepackage{multirow,array}
\usepackage{hyperref}
 \usepackage{graphicx}
%\usepackage{ipe}
%\input{thmstyle-ger}

\parindent0mm
\sloppy

% Basic data
\newcommand{\VORLESUNG}{Deskriptive Statistik für Soziologinnen und Soziologen}
\newcommand{\STAFF}{Mariana Nold}
\newcommand{\ASSIGNMENT}{4}
\newcommand{\HANDOUT}{Montag, den 22.Mai   2017}
\newcommand{\DELIVER}{bis Freitag, den 2. Juni 2017, Briefkasten des Instituts für Soziologie, in der Nähe der Cafeteria der Carls-Zeiss-Straße (In einem Umschlag, an mich adressiert) \textbf{oder} Mittwoch, Donnerstag und Freitag von 13-15 Uhr im Sekretariat von Frau Prof. Leuze, CZ-Straße 2, R286}
\newcommand{\PRACTICAL}[1]{\marginpar{\tiny {\bf Aufgabe \\ abgeben!} #1}}
\newcommand{\FAUFTRAG}[1]{\marginpar{\tiny {\bf selbst entdeckendes Verstehen} #1}}
\newcommand{\titel}{Bivariate Exploration von  quantitativen und qualitativen 
Merkmalen: Korrelation}
\newcommand{\startwert}{14}

% Arbitrary packages and settings

\newcommand{\N}{\mathbb{N}}
\newcommand{\floor}[1]{\lfloor{#1}\rfloor}
\newcommand{\ceil}[1]{\lceil{#1}\rceil}
\newcommand{\half}[1]{\frac{#1}{2}}
\newcommand{\punkte}[1]{{\small{ }(#1 Punkte)}}
\newcommand{\punkt}[1]{{\small{ }(#1 Punkt)}}

\newcommand{\aufgabe}[1]{\item{\bf #1}}
\newcommand{\hinweis}{{\em Hinweis}}

\begin{document}
% Document title

\begin{center}
\ASSIGNMENT{}. Aufgabenblatt vom \HANDOUT{} zur Vorlesung 
\vspace*{0.5cm}

%FT S149 + 156
{\Large \VORLESUNG{}}
%\PRACTICAL{}
(\STAFF{}) 


\vspace*{0.5cm}
{\textbf{Thema:} \titel{}\\}
\vspace*{0.2cm}

{\small Abgabe: \DELIVER{}}
\vspace*{1cm}
\end{center}

\begin{enumerate}\addtocounter{enumi}{\startwert}




\aufgabe{Wortschatz von Kindern} \punkte{20} \PRACTICAL{}\\
(in Anlehnung an: Fahrmeir et al, Statistik Der Weg zur Datenanalyse, S .151)\\
Bei fünf zufällig ausgewählten Kindern wurden der Körpergröße  $X$  in cm und die 
Wortschatz $Y$  gemessen. Dabei erfolgte die Messung des Wortschatzes
über die Anzahl der verschiedenen Wörter, die die Kinder in einem Aufsatz über die 
Ergebnisse in ihren Sommerferien benutzten. Nehmen wir an,
wir hätten folgende Daten erhalten:

 \begin{table}[h]
 \centering 
\begin{tabular}{|r|r|r|r|r|r|}
  \hline
  Kind   $i$     &   $1$    & $2$   & $3$   & $4$   & $5$ \\ \hline
  Körpergröße  $x_{i}$ & $130$ & $112$ & $108$ & $114$ & $136$  \\ \hline
  Wortschatz  $y_{i}$ & $37$& $30$ & $20$&  $28$ & $35$ \\ \hline
 % Alter  $z_{i}$ & $12$& $7$ & $6$&  $7$ & $13$ \\ \hline
  \end{tabular}
 \caption{Die Körpergröße  $X$ und der Wortschatz $Y$ in cm gemessen von $5$
 zufällig ausgewählten Kindern. \label{tab1}}
 \end{table}  
 
\begin{enumerate}
\item{Zeichnen Sie ein Streudiagramm.\\
\textbf{Lösung:}\\
\textit{Siehe Abbildung \ref{fig_kw}.}
} 

%  \begin{figure}[ht]
% 	\centering
% 	      \includegraphics[width=0.75\textwidth]{scatter_15.pdf}
% 	      \caption{Körpergröße in cm und Wortschatz von fünf Kindern. \label{fig_kw}}
% 	\end{figure}
\item{Erklären Sie an Hand dieses Beispiels was eine
\begin{itemize}
\item{positive bzw. negative lineare Korrelation} %\punkte{3}
\item{positive bzw. negative monotone Korrelation} %\punkte{2}
\end{itemize}
inhaltlich bedeuten.\\
\textbf{Lösung:}\\
\textit{Eine Korrelation beschreibt die Beziehung zwischen zwei Merkmalen. Man kann als 
Synonym für Korrelation den Begriff Wechselbeziehung verwenden. Es bedeutet,
dass einen Einfluss von einem Merkmal $X$ auf eine anderes Merkmal $Y$ gibt,
wenigstens in eine Richtung. Ist der Einfluss nur in eine Richtung, dann liegt
eine Spezialfall einer Wechselbeziehung vor. Dann sind die Rollen des abhängigen
und des unabhängigen Merkmals klar verteilt. Das abhängige Merkmal wird vom unabhängigen
Merkmal beeinflusst.\\ Die Korrelation mach keine Aussage darüber,
ob die Rollen klar verteilt sind.
Ein \textbf{positive} Korrelation der Merkmale $X$ und $Y$ bedeutet, einfach gesagt,
\glqq je mehr $X,$ desto mehr $Y$.\grqq Diese Aussage beschreibt eine mittlere Tendenz.
Eine positive Korrelation zwischen Körpergröße und Wortschatz bedeutet,
dass man im Mittel beobachtet, dass größere Kinder einen höheren Wortschatz haben,
oder umgekehrt, Kinder mit größerem Wortschatz größer sind. Die Korrelation
macht keine Aussage über die Richtung des Zusammenhangs.\\
Analog beschreibt eine \textbf{negative} Korrelation zwischen zwei Merkmalen $X$ und $Y$
eine negative mittlere Tendenz. Wenn z. B. mit zunehmender Zeit, seit 
Beginn der Bearbeitung einer Aufgabe die Konzentration sinkt, dann sind die
Bearbeitungsdauer $X$ und die Konzentrationsleistung $Y$ negativ korreliert.
In diesem Beispiel ist inhaltlich klar, dass die Bearbeitungsdauer ursächlich ist
für die fallende Konzentration. Die Korrelation selbst beschreibt eine
ungerichtete Wechselbeziehung. \\
In folgendem Artikel wir Korrelation
am Beispiel gemessener Intelligenz und der Schulleistung diskutiert und erklärt:\\
\url{http://www.zeit.de/1974/44/was-ist-eine-korrelation}
}
} 
\item{Ist die folgende Aussage falsch oder richtig: Es ist im Allgemeinen möglich,
dass ein positiver linearer Zusammenhang vorliegt, aber kein positiver
monotoner Zusammenhang.\\
\textbf{Lösung:}
\textit{Diese Aussage ist falsch. Wenn ein positiver linearer Zusammenhang vorliegt,
dann ist dieser Zusammenhang auch monoton. Jede Gerade mit positiver 
Steigung ist eine monoton wachsende Funktion. Umgekehrt gilt allerdings,
ein positiver monotoner Zusammenhang braucht nicht linear zu sein. 
}
} 
\item{Schreiben Sie die Tabelle der Ränge von $X$ und $Y,$
berechnen die den Rangkorrelationskoeffizient $r^{SP}_{XY}$ und interpretieren Sie
diesen Wert.\\
\textbf{Lösung:}\\
\textit{Siehe Tabelle \ref{tab2_sol}. 
Die Formel für den Rangkorrelationskoeffizient ist:
\begin{equation}
\label{sp}
r^{SP}_{XY}= 
  \frac{\sum_{i=1}^{n}(rg(x_{i})-\bar{rg}_{X})\cdot(rg(y_{i})-\bar{rg}_{Y})}
  		{\sqrt{ \sum_{i=1}^{n} (rg(x_{i})-\bar{rg}_{X})^2}\cdot 
  		\sqrt{ \sum_{i=1}^{n}(rg(y_{i})-\bar{rg}_{Y})^2}},
\end{equation}
dabei gilt:
$$
\bar{rg}_{X}=\frac{1}{n} \sum_{i=1}^{n} rg(x_{i}) = \frac{1}{n} \sum_{i=1}^{n} i = (n+1)/2
$$
und 
$$
\bar{rg}_{Y}=\frac{1}{n} \sum_{i=1}^{n} rg(y_{i}) = \frac{1}{n} \sum_{i=1}^{n} i = (n+1)/2
$$
Für die mittleren Ränge $\bar{rg}_{X}$ und $\bar{rg}_{Y}$ ergibt sich also in unserem Beispiel:
$\bar{rg}_{X}=\bar{rg}_{Y}=(5+1)/2=3$
}
 \begin{table}[h]
 \centering 
\begin{tabular}{|r|r|r|r|r|r|}
  \hline
  Kind   $i$     &   $1$    & $2$   & $3$   & $4$   & $5$ \\ \hline
   $rg(x_{i})$ & $4$ & $2$ & $1$ & $3$ & $5$  \\ \hline
  $rg(y_{i})$   & $5$& $3$ & $1$&  $2$ & $4$ \\ \hline
   $rg(x_{i})-\bar{rg}_{X}$ & $1$ & $-1$ & $-2$ & $0$ & $2$  \\ \hline
  $rg(y_{i})-\bar{rg}_{Y}$   & $2$& $0$ & $-2$&  $-1$ & $1$ \\ \hline
   $(rg(x_{i})-\bar{rg}_{X})(rg(y_{i})-\bar{rg}_{Y})$   & $2$& $0$ & $4$&  $0$ & $2$ \\ \hline
  $(rg(x_{i})-\bar{rg}_{X})^2$   & $1$ & $1$ & $4$ & $0$ & $4$  \\ \hline
  $(rg(y_{i})-\bar{rg}_{Y})^2$   & $4$& $0$ & $4$&  $1$ & $1$ \\ \hline
  \end{tabular}
 \caption{Die Ränge der Körpergröße   $X$  und  des Wortschatzes $Y$ 
 von fünf
 zufällig ausgewählten Kindern. Erweiterte Tabelle zur Berechnung des Rangkorrelationskoeffizient.
 \label{tab2_sol}}
 \end{table}  
 
 \textit{Mit Hilfe der Tabelle kann man die Formel \eqref{sp} in drei Teile zerlegen. Im Zähler steht die 
 Summe über die Produkte die in Zeile 6 der Tabelle berechnet sind. Summiert man diese Summe auf,
 so erhält man den Wert $8.$ Der Nenner besteht aus zwei Teilen, die miteinander multipliziert 
 werden. Um den ersten Teil zu berechnen, summiert man die 7. Zeile auf und zieht dann die Wurzel.
 Es ergibt sich der Wert $\sqrt{10} \approx 3.162.$ Entsprechend erhält man den zweiten Teil aus 
 Zeile 8. Der Werte ist $\sqrt{10} \approx 3.162.$ Insgesamt ergibt sich also
 $$r_{X,Y}^{SP} =\frac{8}{\sqrt{10}\cdot \sqrt{10}}=0.8.$$ Dieser Wert spricht für einen ziemlich starken
 monotonen Zusammenhang. Er macht keine Aussage darüber, ob dieser Zusammenhang
 linear ist, oder nicht.
 }
}
\item{Berechnen Sie nun  den Korrelationskoeffizienten nach Pearson $r_{X,Y}$
und interpretieren Sie diesen Wert.\\
\textbf{Lösung:}\\
\textit{Die Formel für den Korrelationskoeffizienten nach Pearson $r_{X,Y}$ ist:
\begin{equation}
\label{pear}
r_{X,Y}= \frac{\hat{\sigma}_{X,Y}}{\hat{\sigma}_{X}\cdot \hat{\sigma}_{Y}}=
  \frac{\sum_{i=1}^{n}(x_{i}-\bar{x}) \cdot
  		(y_{i}-\bar{y})}{\sqrt{ \sum_{i=1}^{n} (x_{i}-\bar{x})^{2}}\cdot \sqrt{ \sum_{i=1}^{n} (y_{i}-\bar{y})^2}}
\end{equation}
Man erkennt, dass der Aufbau der gleiche ist, wie im Rangkorrelationskoeffizienten. Der Unterschied
ist: Man rechnet hier nicht mit den Rängen der Beobachtungen, sondern mit den 
Beobachtungen selbst. Man berechnet $\bar{x}=120$ und $\bar{y}=30.$ Entsprechend bildet man die Tabelle:
 \begin{table}[h]
 \centering % 49   0 100   4  25
\begin{tabular}{|r|r|r|r|r|r|}
  \hline
  Kind   $i$     &   $1$    & $2$   & $3$   & $4$   & $5$ \\ \hline
    Körpergröße  $x_{i}$ & $130$ & $112$ & $108$ & $114$ & $136$  \\ \hline
  Wortschatz  $y_{i}$ & $37$& $30$ & $20$&  $28$ & $35$ \\ \hline
$x_{i}-\bar{x}$ & $10$ & $-8$ & $-12$ & $-6$ & $16$  \\ \hline
  $y_{i}-\bar{y}$   & $7$& $0$ & $-10$&  $-2$ & $5$ \\ \hline
   $(x_{i}-\bar{x})(y_{i}-\bar{y})$   & $70$& $0$ & $120$&  $12$ & $80$ \\ \hline
  $(x_{i}-\bar{x})^2$   & $100$ & $64$ & $144$ & $36$ & $256$  \\ \hline
  $(y_{i}-\bar{y})^2$   & $49$& $0$ & $100$&  $4$ & $25$ \\ \hline
  \end{tabular}
 \caption{Die Körpergröße   $X$  und  des Wortschatzes $Y$ 
 von fünf
 zufällig ausgewählten Kindern. Erweiterte Tabelle zur Berechnung des Korrelationskoeffizient.
 \label{tab3_sol}}
 \end{table}  
 Wie oben berechnet man die Summer der 6. Zeile um den Zähler zu bestimmen. Man erhält $282.$
 Für den Zähler erhält man beruhend auf der 7. Zeile. Es ergibt sich $\sqrt{600}.$ Den zweiten 
 Teil des Produkts im Zähler erhält man mit Zeile 9. Es ergibt sich $\sqrt{178.}$ Zusammenfassend 
 erhält man 
 $$
 r_{X,Y}=\frac{282}{\sqrt{600}\cdot \sqrt{178}}=0.863.
 $$
 Diese Zahl spricht für einen starken linearen Zusammenhang. 
}
}
 \begin{table}[h]
 \centering 
\begin{tabular}{|r|r|r|r|r|r|}
  \hline
  Kind   $i$     &   $1$    & $2$   & $3$   & $4$   & $5$ \\ \hline
  Körpergröße  $x_{i}$ & $130$ & $112$ & $108$ & $114$ & $136$  \\ \hline
  Wortschatz  $y_{i}$ & $37$& $30$ & $20$&  $28$ & $35$ \\ \hline
  Alter  $z_{i}$ & $12$& $7$ & $6$&  $7$ & $13$ \\ \hline
  \end{tabular}
 \caption{Die Körpergröße   $X$  in cm gemessen, der Wortschatz $Y$
 und das Alter $Z$ von fünf
 zufällig ausgewählten Kindern. \label{tab2}}
 \end{table}  
 \item{Die Tabelle \ref{tab2} enthält zusätzlich das Alter $Z$ der Kinder,
berechnen Sie jeweils $r_{YZ}$ und $r_{XZ}$ und interpretieren
Sie auch diese Werte.\\
\textbf{Lösung:}\\
\textit{Wie oben in Teilaufgabe e, berechnet man $r_{X,Z}=0.995$ und $r_{Y,Z}=0.867.$
Man erkennt, dass der Körpergröße $X$ und dem Alter $Z$ ein sehr deutlicher linearer 
Zusammenhang besteht. Auch zwischen dem Alter $Z$ und dem Wortschatz $Y$ besteht ein
deutlicher Zusammenhang.
}
}

\end{enumerate}

\newpage

\aufgabe{Kreuztabellen interpretieren: Habilitationsdichte}   \punkte{11} \PRACTICAL{}\\

Die Habilitation ist die höchstrangige Hochschulprüfung in Deutschland durch Anfertigung einer wissenschaftlichen Arbeit.
In einer Untersuchung zur Habilitationsdichte an deutschen Hochschulen wurden u. a. die Merkmale Geschlecht und Habilitationsfach erhoben. In Tabelle 
\ref{tab4} ist - nach Fächern aufgeschlüsselt- zusammengefasst, wieviele Habilitationen im Jahre 2015 erfolgreich abgeschlossen wurden 
(Quelle: Statistisches Bundesamt) Hier stellt sich die Frage, ob die Habilitationsdichte in den einzelnen Fächern im Jahr 2015
geschlechtsspezifisch ist, d. h. man interessiert sich dafür, ob zwischen den Merkmalen Geschlecht (=:Y) und Habilitationsfach (=:X) ein Zusammenhang besteht.


   \[
   \text{Ausprägungen} \: X  \overset{\wedge}{=} \left\{\begin{array}{ll} % K, &  \text{\glqq keine Ausbildung\grqq} \\
         a_1, &  \text{ Geisteswissenschaften} \\
         a_2, &  \text{ Rechts-,Wirtschafts-,Sozialwiss.} \\
         a_3, &  \text{ Mathe u. Naturwiss.} \\
         a_4, &  \text{Human-,Gesundheitswiss.} \\
         a_5, &  \text{ übrige Fächer} \\
        \end{array}\right.
  \]
  
     \[
    \text{Ausprägungen} \:  Y  \overset{\wedge}{=} \left\{\begin{array}{ll}  
         b_1, &  \text{  Frauen} \\
         b_2, &  \text{ Männer } \\
        \end{array}\right.
  \]

\begin{table}[h]
\centering
\begin{tabular}{l l|c|c|c|c|c|c}
\multicolumn{2}{c}{}&\multicolumn{4}{c}{$X$}& &$\sum$\\
%\cline{3-7}
\multicolumn{2}{c}{}&$a_{1}$& $a_{2}$& $a_{3}$& $a_{4}$ & $a_{5}$&\\
\cline{3-7}
\multirow{2}{*}{$Y$}& $b_{1}$ & $77 $ & $62$ &  $ 66$ & $225$ & $32$&\\
\cline{2-7}
& $b_{2}$ & $159$ & $139$ &  $181$ & $571$ & $115$&\\
\cline{2-7}
& $\sum$ &  &  &   &  & &\\
%\caption{\label{tab4}}
\end{tabular}
\caption{Habilitationen im Jahre 2015 erfolgreich abgeschlossen wurden nach Fächern und Geschlecht aufgeschlüsselt\label{tab4}.}
\end{table}
\begin{enumerate}
\item{Ergänzen sie die fehlenden Randhäufigkeiten.\\
\textbf{Lösung:}\\
\textit{Siehe Tabelle \ref{tab4_sol}}
\begin{table}[h]
\centering
\begin{tabular}{l l|c|c|c|c|c|c}
\multicolumn{2}{c}{}&\multicolumn{4}{c}{$X$}& &$\sum$\\
%\cline{3-7}
\multicolumn{2}{c}{}&$a_{1}$& $a_{2}$& $a_{3}$& $a_{4}$ & $a_{5}$&\\
\cline{3-7}
\multirow{2}{*}{$Y$}& $b_{1}$ & $77 $ & $62$ &  $ 66$ & $225$ & $32$&462\\
\cline{2-7}
& $b_{2}$ & $159$ & $139$ &  $181$ & $571$ & $115$&1165\\
\cline{2-7}
& $\sum$ & $236$ & $201$ & $247$   & $796$  & $147$ & $1627$\\
%\caption{\label{tab4}} %236 201 247 796 147
\end{tabular}
\caption{Habilitationen im Jahre 2015 erfolgreich abgeschlossen wurden nach Fächern und Geschlecht aufgeschlüsselt\label{tab4_sol} mit Randhäufigkeiten.}
\end{table}
} 
\item{Berechnen Sie die Randverteilungen (= marginale relative Häufigkeit).\\
\textbf{Lösung:}\\ Die Randverteilung des Merkmals Geschlecht (Y) ist
$f_{1,\bullet}=0.284, f_{2,\bullet}=0.716.$ Die Randverteilung des Merkmals Habilitationsfach (X)
ist:\\ $f_{\bullet,1}=0.145, f_{\bullet,2}=0.124, f_{\bullet,3}= 0.152, f_{\bullet,4}= 0.489,
f_{\bullet,5}= 0.090$
}


\item{Wie hoch ist der Anteil der Frauen, die im Jahr $2015$ eine Habilitation abgeschlossen
haben.\\
\textbf{Lösung:}\\
\textit{Der Anteil beträgt $\frac{462}{1627}=0.284.$ Von allen die $2015$ eine Habilitation
abgeschlossen haben, sind $28.4\%$ Frauen.}
}
\item{Wie hoch ist der Anteil an Habilitationen aus dem Fachbereich \glqq Mathematik
und Naturwissenschaften\grqq?\\
\textbf{Lösung:}\\
\textit{Der Anteil ist $f_{\bullet,3}= 0.152.$ Damit sind $15.2\%$ der Habilitationen aus
diesem Fachbereich. }
}
\item{Berechnen Sie die  bedingten Verteilungen (= bedingte 
relative Häufigkeit) gegeben dem Fachbereich
und interpretieren Sie das Ergebnis. \\
\textbf{Lösung:}\\
\textit{Siehe Tabelle \ref{tab4_sol_hab}. Die Tabelle zeigt wie sich in den einzelnen Fachbereichen
die Anteile an Frauen und Männern darstellen. Ein Vergleich mit der Randverteilung zeigt,
wo man mehr bzw. weniger Frauen als mit Bezug auf die Randverteilung erwartet, findet.
In den Geisteswissenschaften ist der Frauenanteil am höchsten.}
\begin{table}[h]
\centering
\begin{tabular}{l l|c|c|c|c|c|c}
\multicolumn{2}{c}{}&\multicolumn{4}{c}{$X$}& &$$\\
%\cline{3-7}
\multicolumn{2}{c}{}&$a_{1}$& $a_{2}$& $a_{3}$& $a_{4}$ & $a_{5}$&\\
\cline{3-7}
\multirow{2}{*}{$Y$}& $b_{1}$ & 0.326& 0.308& 0.267& 0.283& 0.218& 0.284\\
\cline{2-7}
& $b_{2}$ & 0.674 & 0.692 & 0.733& 0.717& 0.782&0.716\\
\cline{2-7}
&  & $1.0$ & $1.0$ & $1.0$   & $1.0$  & $1.0$ & $1.0$\\
%\caption{\label{tab4}} %236 201 247 796 147
\end{tabular}
\caption{Bedingte Verteilung des Geschlechts ($Y$) gegeben dem
Habilitationsfach ($X$)\label{tab4_sol_hab} .}
\end{table}
} 
\item{Berechnen Sie die  bedingten Verteilungen  (= bedingte 
relative Häufigkeit) gegeben das Geschlecht
und interpretieren Sie das Ergebnis. \\
\textbf{Lösung:}\\
\textit{Siehe Tabelle \ref{tab4_sol_sex}. Die Tabelle zeigt für die Gruppe der Männer
und die Gruppe der Frauen jeweils, wie hoch der Anteil der Habilitationen in
den einzelnen Fachbereichen ist. Unter den Frauen wurden die meisten
Habilitationen in den Human- und Gesundheitswissenschaften abgeschlossen ($48.7\%$).
In der Gruppe der Männer liegt dieser Fachbereich ebenfalls ganz vorne mit $49.0\%$}
\begin{table}[h]
\centering
\begin{tabular}{l l|c|c|c|c|c|c}
\multicolumn{2}{c}{}&\multicolumn{4}{c}{$X$}& &$$\\
%\cline{3-7}
\multicolumn{2}{c}{}&$a_{1}$& $a_{2}$& $a_{3}$& $a_{4}$ & $a_{5}$&\\
\cline{3-7}
\multirow{2}{*}{$Y$}& $b_{1}$ & 0.167 & 0.134 & 0.143 & 0.487 &0.069& 1.0\\
\cline{2-7}
& $b_{2}$ & 0.136& 0.119& 0.155& 0.490& 0.099& 1.0\\
\cline{2-7}
&  & 0.145 & 0.124 & 0.152 & 0.489 & 0.090 &1.0\\
%\caption{\label{tab4}} %236 201 247 796 147
\end{tabular}
\caption{Bedingte Verteilung des Habilitationsfach ($X$) gegeben dem
Geschlechts ($Y$)\label{tab4_sol_sex}.}
\end{table}
} 
\end{enumerate}

\newpage
\aufgabe{Die Korrelation mit STATA berechnen}


Das Streudiagramm \ref{fig1} zeigt die Spielbewertung eines neuen Spiels aufgetragen
auf der Ordinate und die Mathe-Punkte (hier simuliert, nicht aus den PISA-Daten)
von $300$ Schülerinnen bzw. Schülern. Sie finden den entsprechenden Datensatz
auf den Rechnern im Methoden-Labor im Ordner \texttt{Methoden/Statistik.}
% https://statistics.laerd.com/stata-tutorials/spearmans-correlation-using-stata.php
% https://statistics.laerd.com/stata-tutorials/pearsons-correlation-using-stata.php
%  \begin{figure}[ht]
% 	\centering
% 	      \includegraphics[width=0.75\textwidth]{scatter_curve2.pdf}
% 	      \caption{Spielfreude der Schülerinnen und Schüler versus Mathe-Score beruhend auf 
% 	      simulierten Daten. \label{fig1}}
% 	\end{figure}
 	\begin{enumerate}
 	\item{Öffnen Sie den Datensatz und geben Sie den Befehl \texttt{summarize} in das \texttt{command}-Fenster
 	ein. Interpretieren Sie die von \texttt{STATA} erzeugte Tabelle.}
 	\item{Geben Sie die Befehle \texttt{graph box x} und \texttt{graph box y} ein und interpretieren Sie
 	die entsprechenden Boxplots.}
 	\item{Erzeugen Sie mit dem Befehl \texttt{scatter y x} das Streudiagramm.}
 	\item{Berechnen Sie mit \texttt{pwcorr x y} den Korrelationskoeffizient nach Pearson.
 	Wie ändert sich das Ergebnis, wenn Sie den Befehl \texttt{pwcorr y x} eingeben.
 	Interpretieren Sie diese Veränderung inhaltlich.}
 	\item{Berechnen Sie mit Hilfe des Befehls \texttt{ spearman x y} den Wert des Rangkorrelationskoeffizienten.}
 	\end{enumerate}
%http://stackoverflow.com/questions/10711395/spearman-correlation-and-ties
%http://www.crashkurs-statistik.de/


\end{enumerate}

\end{document}
