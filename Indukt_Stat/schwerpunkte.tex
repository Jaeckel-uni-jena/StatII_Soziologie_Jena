% \documentclass[a4paper,12pt]{article}
% \usepackage[utf8]{inputenc}
% \usepackage[ngerman]{babel} 
% %opening
% \title{}
% %\author{}
% \date{}
\documentclass[a4paper,fontsize=16pt]{article}
%\documentclass[a4paper,fontsize=13pt]{scrartcl}
%\definecolor{mygreen}{cmyk}{0.82,0.11,1,0.25}
\usepackage[utf8]{inputenc}
\usepackage{amsmath}
\usepackage{mathtools}
\usepackage{hyperref}
\usepackage[ngerman]{babel} 
\usepackage{textcomp} 
\usepackage{color}
\usepackage{xcolor}
\usepackage[margin=1.55cm
  %,showframe% <- only to show the page layout
]{geometry}
\usepackage{rotating}
\usepackage{booktabs} 
\usepackage{lscape}
\usepackage{float}
%opening
\title{Schwerpunkte: Induktive Statistik WS 17/18}
%\subtitle{Simulation und Auswertung der simulierten Daten in Übung und Tutorium mit Hilfe von \texttt{R}}
%\author{Mariana Nold}
\date{\vspace{-10ex}}
\begin{document}

\maketitle

Termine: 
16./23.10.17
13./27.11.17
11.12.17
08.01.18
22.01.18

\section{Schwerpunkte}
\begin{enumerate}
\item{\colorbox{yellow}{Probabilistische Gesetze: Was ist das und welche Rolle spielen sie in der klassischen Inferenz?}}
\item{\colorbox{pink}{Grundlagen der klassischen Schätz- und Testtheorie}}
\item{\colorbox{green}{Statistische Modelle: Multiple parametrische Datenanalyse}}
\end{enumerate}

\section{Statistische Methoden}
\begin{enumerate}
\item{Daten selbst analysieren: Thüringen-Monitor 2015 und Pisa-Daten mit STATA}
\item{Statistische Informationen nutzen und sachadäquat interpretieren: STATA-Output und Information aus Presse oder Fachliteratur (insbesondere Veröffentlichungen über Thüringen-Monitor, Pisa-Daten oder KIGGS-Daten)}
\item{Statistische Ergebnisse verständlich kommunizieren, wesentliche Aussagen in einfachen Worten ausdrücken können}
\end{enumerate}

\section{Inhaltliche Gliederung}
\begin{enumerate}
   \item{ \textbf{\colorbox{yellow}{Probabilistische Gesetze} (16.10.17)}
   \begin{itemize}
      \item{Wiederholung: Wozu braucht man Statistik?}
      \item{Warum das prob. Modell in der Soziologie? (Beispiel: Thüringen-Monitor oder Tagespresse)}
      \item{Was ist eine stoch. bzw. prob. Aussage? (Ist Unterscheidung sinnvoll? S. 2, Schumann)}
      \item{Beispiel: Zufall und Klausureinsicht}
      \item{Motivation Binomialverteilung: Wie viele Personen kommen zur Klausureinsicht?}
      \item{Summe von Bernoulli-Variablen: Jede Person entscheidet unabhängig, ob sie kommt}
      \item{Zufallsstichprobe und Totalerhebung}
       \item{\colorbox{gray}{Literatur:} 
       \begin{itemize}
       \item{ Gehring+Weins: Abschnitt 1.2.4 (S.9)}
       \item{Mittag: Kap. 11}
       \item{Bortz Kap. 6.1}
       \end{itemize}}
   \end{itemize}
   }
   \item{ \textbf{\colorbox{yellow}{So tickt die klassische Inferenz am Beispiel der Biomailverteilung}(23.10.17)
  }
     
       \begin{itemize}
       \item{Definition: diskrete Wahrscheinlichkeitsverteilung}
       \item{Die Bernoulliverteilung als  diskrete Wahrscheinlichkeitsverteilung : Erwartungswert und Varianz}
         \item{Die Binomialverteilung als diskrete Wahrscheinlichkeitsverteilung : Erwartungswert und Varianz}
         \item{Intuitiver Schätzer für die Wahrscheinlichkeit dass jemand zur Klausureinsicht kommt}
         \item{Konfidenzintervall für die Wahrscheinlichkeit dass jemand zur Klausureinsicht kommt}
         \item{Wie viele Personen werden erwartet? Wir große ist die Streuung?}
         \item{Weiteres Beispiel für diskrete Verteilung: Poisson-Verteilung (Anzahl der Flüchtlinge pro Tag \dots)}
          \item{\colorbox{gray}{Literatur:} 
       \begin{itemize}
       \item{ Ludwig-Mayerhofer: Kap. 4.1 }
       \item{Mittag: Kap.11}
       \end{itemize}}
         \item{\textbf{Übung:} So wenig Fleisch im Gulasch: Kann das Zufall sein?}
    %     \item{Intuitives Verständnis für das Gesetz der großen Zahlen}
         \end{itemize}
      } 
 
%      \item{ \textbf{Verteilung des Mittelwerts mit wachsender Anzahl von Versuchen}
%        \begin{itemize}
%           \item{Mittelwert als Schätzer} % 
%           \item{Intuitives Verständnis für Gesetz der großen Zahlen}
%         \end{itemize}
%    }
   
    \item{\textbf{\colorbox{pink}{Punktschätzer, Konfidenzintervall und Hypothesentest}(13.11.17)}
     \begin{itemize}
          \item{Wiederholung: Was ist eine Konfidenzintervall?} % 
          \item{Wiederholung: Parameter der Normalverteilung}
          \item{Definition: stetige Wahrscheinlichkeitsverteilung}
          \item{Konfidenzintervall für $\mu,$ wenn $\sigma$ bekannt. (Beispiel: Pisa)}
          \item{z-Test für $\mu$, Zusammenhang mit Konfidenzintervall}
          \item{$\alpha$ und $\beta$ Fehler}
          \item{\colorbox{gray}{Literatur:} 
          \begin{itemize}
       \item{ Gehring+Weins: Kap. 10.3 +11, S. 235 ff (ohne 10.4)}
       \item{Mittag: Kap. 12.3}
       \item{Diaz- Bone Kap. 7 S.164 ff + S.155 ff }
       \item{Bortz Kap. 6.2 + 6.3} 
       \end{itemize}}
           
     \end{itemize}
   }
   
      \item{\textbf{\colorbox{pink}{Grundlage des statistischen Testens}(27.11.17)}
     \begin{itemize}
          \item{Wiederholung: $\chi^2$-Koeffizient (aus dem letzten Semester)} % 
          \item{Beispiel: Ganz viel im Thüringen-Monitor?}
          \item{Wie groß muss $\chi^2$-Koeffizient sein, damit man nicht mehr vom Zufall ausgeht? ($\chi^{2}-$ Unabhängigkeitstest)}
          \item{Überblick: Arten von stat. Tests}
          \item{Namen von Test, die häufig verwendet werden z. B. Gauss-Test, t- Test}
          \item{\colorbox{gray}{Literatur:} 
          \begin{itemize}
       \item{Mittag: Kap. 15}
       \item{Diaz- Bone Kap. 177 ff }
       \item{Bortz Kap. 7,8,9} 
       \end{itemize}}
     \end{itemize}
   }
  
     
%     \item{\textbf{Grundlage des statistischen Testens }
%     \begin{itemize}
%       \item{Normalverteilung und Mittelwert}
%       \item{Tests über den Mittelwert}
%       \item{Konfidenzintervall für den Mittelwert }
%       \item{Zusammenhang der Interpretation von Konfidenzintervall und Test}
%     \end{itemize}
%   }
%     

  
   \item{\textbf{\colorbox{green}{Bivariate lineare Regression und Varianzanalyse}(11.12.17)}
     \begin{itemize}
          \item{Wiederholung aus Stat 1 } % 
          \item{Warum Regression?}
          \item{Testen und Schätzen in der linearen Einfachregression}
          \item{Varianzanalyse, t-Test und Regression können angewendet werden,
          wenn man ein normalverteiltes Merkmal in zwei Gruppen vergleicht}
           \item{\colorbox{gray}{Literatur:} 
          \begin{itemize}
       \item{ Gehring+Weins: Kap. 8}
       \item{Mittag: S. 245-254}
       \item{Diaz- Bone Kap. 8, Anfang }
       \end{itemize}}
     \end{itemize}
   }
   
   \item{\textbf{\colorbox{green}{Multivariate lineare Regression}(08.01.17)}
     \begin{itemize}
        %  \item{Kovarianz und Korrelation} % 
          \item{Interpretation von Schätzern, Konfidenzintervallen und Tests}
          \item{Interpretation der Regressionsgeraden und Konfidenzintervall der Regressionsgeraden}
          \item{F-Test}
          \item{Multikollinearität}
          \item{\colorbox{gray}{Literatur:} 
          \begin{itemize}
       \item{Diaz- Bone Kap. 8, S. 189 ff }
       \item{Ludwig-Mayerhofer: Kap. 6}
       \item{Fahrmeir, Kneib Lang: Regression (für Details)}
       \end{itemize}}
     \end{itemize}
   }
   
         \item{\textbf{\colorbox{green}{Varianzanalyse und Wiederholung des bisherigen Stoffes}(22.01.17)}
    \begin{itemize}
      \item{Vergleich von Mittelwerten in zwei Gruppen (kurze Wiederholung)}
      \item{Vergleich von Mittelwerten in drei und mehr Gruppen}
      \item{Wiederholung}
       \item{\colorbox{gray}{Literatur:} 
          \begin{itemize}
       \item{Mittag: Kap. 17}
       \item{Ludwig-Mayerhofer: Kap. 5.4}
       \item{Fahrmeir, Tutz : Statistik Kap. 13}
       \end{itemize}}
    \end{itemize}
  }
%  \item{\textbf{Logistische Regression}
%      \begin{itemize}
%        \item{Anwendungsbeispiele }
%        \item{KI und Test für Effekte}
%      \end{itemize}
%    }
  
 % \item{\textbf{Wiederholung}}
  
\end{enumerate}

\textbf{Nicht enthalten}: zentraler Grenzwertsatz und Gesetz der großen Zahlen (insbesondere W'keit als Grenzwert der rel. Häufigkeit) 
% Aufgaben der deskrip. Statistik:

% ist es, die in den Daten einer Stichprobe enthaltene relevante Information in Tabellen, Grafiken und statistischen 
% % Maßzahlen übersichtlich und in einem der Fragestellung angemessenen Format zusammenzufassen.

% In der Drittvariablenkontrolle versucht man
% Effekte zwischen zwei Variablen unter
% Konstanthaltung (Aussschaltung) der
% anderen ins Modell aufgenommenen
% Variablen zu berechnen

\end{document}
